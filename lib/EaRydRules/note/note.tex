\documentclass{article}
\usepackage[colorlinks=true,urlcolor=blue,citecolor=blue,linkcolor=blue]{hyperref} 

\usepackage{amsmath}
\usepackage{multirow}
\usepackage[linesnumbered, ruled, vlined]{algorithm2e}
\usepackage{graphicx}% Include figure files
\usepackage{subcaption}
\usepackage{tabularx}
\usepackage{listings}
\usepackage{amsthm}
\usepackage{cancel}
\usepackage[mode=buildnew]{standalone}
\theoremstyle{definition}
\newtheorem{definition}{Definition}[section]
\newtheorem{theorem}{Theorem}[section]
\newtheorem{corollary}{Corollary}[section]
%\usepackage[linesnumbered, ruled, vlined, algo2e]{algorithm2e}
\lstset{
    basicstyle=\ttfamily\small,
    numberstyle=\scriptsize,
    % numbers=left,
    backgroundcolor=\color{white},
    %backgroundcolor=\color{white},
    %frame=single,
    xleftmargin=2em,
    tabsize=2,
    rulecolor=\color{black},
    %title=\lstname,
    escapeinside={(*}{*)},
    breaklines=true,
    %breakatwhitespace=true,
    %framextopmargin=2pt,
    %framexbottommargin=2pt,
    frame=bt,
    extendedchars=true,
    inputencoding=utf8,
    columns=fullflexible,
    %escapeinside={(*@}{@*)},
}

% the color of power ragers
\newcommand{\blue}[1]{[{\bf  \color{blue}{JG: #1}}]}
\newcommand{\Eq}[1]{Eq.~(\ref{#1})}
\newcommand{\Fig}[1]{Fig.~\ref{#1}}
\newcommand{\ra}[1]{\renewcommand{\arraystretch}{#1}}
\newcommand{\pdv}[2]{{\frac{\partial#1}{\partial#2}}}

\title{Quantum simulation on a random tensor network}
\begin{document}
\maketitle

\section{Differentiating single step time evolution}
The $m$ site Rydberg Hamiltonian is
\begin{equation}
    H_{\rm Rydberg} = \sum_{i,j=1, i>j}^{m} \frac{C}{|r_i-r_j|^6}n_i n_j + \Omega(t)\sum_{i=1}^{m} \frac{1}{2}\sigma_i^x + \Delta(t)\sum_{i=1}^{m} n_i
\end{equation}

For simplicity, we consider the following general representation of a time-space dependent Hamiltonian with $k$ terms
\begin{equation}
    H = \sum_{k=1}^{K} c_k O_k
\end{equation}
where $c_k$ can be dependent on a set of parameters like locations $r_1, r_2, \ldots, r_m$, and pulses $\Omega(t)$ and $\Delta(t)$.

\subsection{The ODE version}
In each step of the ODE solver, it performs the following update
\begin{equation}
    |\psi'\rangle = (1 - iH \Delta t) |\psi\rangle
\end{equation}

We derive the backward rules for the gradients by inspecting the following equations
% \begin{align}
%     &\overline{p} \mathrel{+}= \overline{c_k} \frac{\partial c_k}{\partial p}
% \end{align}

\begin{equation}\label{eq:ad0}
    \begin{split}
    \overline{\mathcal{L}}\delta \mathcal{L} &= \overline{|\psi'\rangle} \circ \delta |\psi'\rangle\\
    &= \sum_k \overline{c_k} \delta c_k + \overline{|\psi\rangle}\circ\delta|\psi\rangle + \overline{\Delta t}\delta \Delta t
    \end{split}
\end{equation}

where $\circ$ is the Hadamard product applied on real numbers,
note a complex number in computer is composed of two real numbers.
The above euqations has a more elegant linear algebra version as the following.

\begin{equation}\label{eq:ad1}
    \begin{split}
    \overline{\mathcal{L}}\delta \mathcal{L} &= \overline{\langle\psi'|} \delta |\psi'\rangle\\
    &= \sum_k \overline{c_k} \delta c_k + \overline{\langle\psi|}\delta|\psi\rangle + \overline{\Delta t}\delta \Delta t
    \end{split}
\end{equation}
where we have used $\langle\psi|$ to represent the hermitian conjugate of $|\psi\rangle$.

\begin{equation}\label{eq:ad2}
    \delta|\psi'\rangle = -i\sum_k \delta c_k O_k \Delta t |\psi\rangle - i H \delta \Delta t |\psi\rangle + (1-iH\Delta t) \delta |\psi\rangle
\end{equation}

By observing \Eq{eq:ad1} and \Eq{eq:ad2}, one can see
\begin{align}
    \overline{\langle\psi|} &= \overline{\langle\psi'|}(1-iH\Delta t)\\
    \overline{c_k} &= \Re\left[-i\Delta t\overline{\langle\psi'|} O_k|\psi\rangle\right]\\
    \overline{\Delta t} &= \Re\left[-i\overline{\langle\psi'|} H|\psi\rangle\right]
\end{align}

After a step, a normalization procedure might be called on the wave functions,
this is trivial so that we do not discuss it at this stage.

\subsection{The \texttt{expmv} version}
It is harder to differentiate the time evolution directly than differentiating an ODE step.

\section{How to reverse the time evolution}
\begin{enumerate}
    \item Since the time evolution of a Hamiltonian is sympletic, one can reverse it by doing inverse time evolution.
    \item For the cases reversibility is not guarented, one can use treeverse algorithm.
\end{enumerate}

\bibliographystyle{plain}
\bibliography{refs}
\end{document}
